\section{Contributions}
In this thesis, I attempt to synthesize the lessons learned from analysing public neuroimaging data with open source software. To this effect, I participated in an international collaboration to create an \ac{MEG} standard for  \ac{BIDS}~\citep{galan2017meg}. I wrote the validator which helped create the MEG-BIDS compatible example datasets. As a contributor to MNE~\citep{gramfort2013meg}, I led an effort to write a tutorial paper which reanalyzes the Faces dataset~\citep{wakeman2015multi} for a reproducible group study. In the backdrop of the reproducibility and data sharing movement described in Sections~\ref{sec:reproducibility_crisis} and \ref{sec:intro_datasharing}, we started automating our pipelines which led us to develop a fully automated algorithm for artifact rejection and repair~\citep{jas2016automated, jas2017autoreject}. Finally, we develop algorithms to learn new undiscovered motifs automatically from neural time series data~\citep{jas2017learning}. 

The thesis is organized by chapters to highlight these four main  contribution areas: data sharing, reproducibility, automation for artifact detection, and automated data-driven motif discovery. Each chapter contains the text from the original paper (which has been minimally edited in some parts) prefaced by a 1-2 page description of the context surrounding the work.

\subsection*{Journal publications}
\bibentry{jas2017autoreject}\ \\ \\
\bibentry{jas2017mne}\ \\ (Pending revision at \emph{Frontiers in Neuroscience, Brain Imaging Methods})\ \\ \\
\bibentry{niso2018meg}

\subsection*{Conference publications}
\bibentry{jas2016automated}\ \\ \\
\bibentry{jas2017learning}

\subsection*{Workshop papers}
\bibentry{dengemann2015conc}\

\subsection*{Open source implementations}

\url{http://autoreject.github.io/} \\
\url{http://alphacsc.github.io/} \\
\url{http://mne-tools.github.io/mne-biomag-group-demo/}\\
\url{https://jasmainak.github.io/bids-validator/}

\subsection*{Datasets}

\url{https://openfmri.org/dataset/ds000248/}
